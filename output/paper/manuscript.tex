% Options for packages loaded elsewhere
\PassOptionsToPackage{unicode}{hyperref}
\PassOptionsToPackage{hyphens}{url}
\PassOptionsToPackage{dvipsnames,svgnames,x11names}{xcolor}
%
\documentclass[
  letterpaper,
  DIV=11,
  numbers=noendperiod]{scrartcl}

\usepackage{amsmath,amssymb}
\usepackage{iftex}
\ifPDFTeX
  \usepackage[T1]{fontenc}
  \usepackage[utf8]{inputenc}
  \usepackage{textcomp} % provide euro and other symbols
\else % if luatex or xetex
  \usepackage{unicode-math}
  \defaultfontfeatures{Scale=MatchLowercase}
  \defaultfontfeatures[\rmfamily]{Ligatures=TeX,Scale=1}
\fi
\usepackage{lmodern}
\ifPDFTeX\else  
    % xetex/luatex font selection
\fi
% Use upquote if available, for straight quotes in verbatim environments
\IfFileExists{upquote.sty}{\usepackage{upquote}}{}
\IfFileExists{microtype.sty}{% use microtype if available
  \usepackage[]{microtype}
  \UseMicrotypeSet[protrusion]{basicmath} % disable protrusion for tt fonts
}{}
\makeatletter
\@ifundefined{KOMAClassName}{% if non-KOMA class
  \IfFileExists{parskip.sty}{%
    \usepackage{parskip}
  }{% else
    \setlength{\parindent}{0pt}
    \setlength{\parskip}{6pt plus 2pt minus 1pt}}
}{% if KOMA class
  \KOMAoptions{parskip=half}}
\makeatother
\usepackage{xcolor}
\setlength{\emergencystretch}{3em} % prevent overfull lines
\setcounter{secnumdepth}{5}
% Make \paragraph and \subparagraph free-standing
\ifx\paragraph\undefined\else
  \let\oldparagraph\paragraph
  \renewcommand{\paragraph}[1]{\oldparagraph{#1}\mbox{}}
\fi
\ifx\subparagraph\undefined\else
  \let\oldsubparagraph\subparagraph
  \renewcommand{\subparagraph}[1]{\oldsubparagraph{#1}\mbox{}}
\fi


\providecommand{\tightlist}{%
  \setlength{\itemsep}{0pt}\setlength{\parskip}{0pt}}\usepackage{longtable,booktabs,array}
\usepackage{calc} % for calculating minipage widths
% Correct order of tables after \paragraph or \subparagraph
\usepackage{etoolbox}
\makeatletter
\patchcmd\longtable{\par}{\if@noskipsec\mbox{}\fi\par}{}{}
\makeatother
% Allow footnotes in longtable head/foot
\IfFileExists{footnotehyper.sty}{\usepackage{footnotehyper}}{\usepackage{footnote}}
\makesavenoteenv{longtable}
\usepackage{graphicx}
\makeatletter
\def\maxwidth{\ifdim\Gin@nat@width>\linewidth\linewidth\else\Gin@nat@width\fi}
\def\maxheight{\ifdim\Gin@nat@height>\textheight\textheight\else\Gin@nat@height\fi}
\makeatother
% Scale images if necessary, so that they will not overflow the page
% margins by default, and it is still possible to overwrite the defaults
% using explicit options in \includegraphics[width, height, ...]{}
\setkeys{Gin}{width=\maxwidth,height=\maxheight,keepaspectratio}
% Set default figure placement to htbp
\makeatletter
\def\fps@figure{htbp}
\makeatother

\usepackage{lineno}
\usepackage{rotating}
\usepackage{float}
\usepackage{typearea}
\KOMAoption{captions}{tableheading}
\makeatletter
\@ifpackageloaded{caption}{}{\usepackage{caption}}
\AtBeginDocument{%
\ifdefined\contentsname
  \renewcommand*\contentsname{Table of contents}
\else
  \newcommand\contentsname{Table of contents}
\fi
\ifdefined\listfigurename
  \renewcommand*\listfigurename{List of Figures}
\else
  \newcommand\listfigurename{List of Figures}
\fi
\ifdefined\listtablename
  \renewcommand*\listtablename{List of Tables}
\else
  \newcommand\listtablename{List of Tables}
\fi
\ifdefined\figurename
  \renewcommand*\figurename{Figure}
\else
  \newcommand\figurename{Figure}
\fi
\ifdefined\tablename
  \renewcommand*\tablename{Table}
\else
  \newcommand\tablename{Table}
\fi
}
\@ifpackageloaded{float}{}{\usepackage{float}}
\floatstyle{ruled}
\@ifundefined{c@chapter}{\newfloat{codelisting}{h}{lop}}{\newfloat{codelisting}{h}{lop}[chapter]}
\floatname{codelisting}{Listing}
\newcommand*\listoflistings{\listof{codelisting}{List of Listings}}
\makeatother
\makeatletter
\makeatother
\makeatletter
\@ifpackageloaded{caption}{}{\usepackage{caption}}
\@ifpackageloaded{subcaption}{}{\usepackage{subcaption}}
\makeatother
\ifLuaTeX
  \usepackage{selnolig}  % disable illegal ligatures
\fi
\IfFileExists{bookmark.sty}{\usepackage{bookmark}}{\usepackage{hyperref}}
\IfFileExists{xurl.sty}{\usepackage{xurl}}{} % add URL line breaks if available
\urlstyle{same} % disable monospaced font for URLs
\hypersetup{
  pdftitle={Some Title},
  pdfauthor={, and },
  colorlinks=true,
  linkcolor={blue},
  filecolor={Maroon},
  citecolor={Blue},
  urlcolor={Blue},
  pdfcreator={LaTeX via pandoc}}

\title{Some Title}
\author{Lorenzo Fabbri\textsuperscript{1,2,*} \and Martine
Vrijheid\textsuperscript{1,2}}
\date{}

\begin{document}
\maketitle
\linenumbers

\textsuperscript{1} Barcelona Institute for Global Health (ISGlobal),
Barcelona, Spain\\
\textsuperscript{2} Pompeu Fabra University, Barcelona, Spain

\textsuperscript{*} Correspondence:
\href{mailto:lorenzo.fabbri@isglobal.org}{Lorenzo Fabbri
\textless{}lorenzo.fabbri@isglobal.org\textgreater{}}

\section*{Abstract}\label{abstract}
\addcontentsline{toc}{section}{Abstract}

\textbf{Background}

\textbf{Objectives}

\textbf{Methods}

\textbf{Results}

\textbf{Discussion}

\newpage

\begin{itemize}
\tightlist
\item
  The title should be less or equal than 300 characters. It should
  indicate the study design, the subject of the paper, information
  regarding exposures and outcomes assessed, and whether the study was
  observational or experimental.
\item
  The suggested length of the abstract is less or equal than 300 words.
\item
  The suggested length is \textless7,000 words, excluding abstract,
  references, tables, figure captions, acknowledgments, and
  Supplementary Material.

  \begin{itemize}
  \tightlist
  \item
    Concise sub-headings should be less or equal than 8 words, and they
    should be used to organize information rather than summarize the
    results.
  \item
    In-text citations with superscript numbers: outside periods and
    commas, but inside colons and semicolons.
  \end{itemize}
\end{itemize}

\section{Introduction}\label{sec-intro}

\subsection{Background and rationale}\label{sec-background}

\begin{itemize}
\tightlist
\item
  Brief review of the literature to summarize current knowledge.

  \begin{itemize}
  \tightlist
  \item
    Acknowledge inconsistencies.
  \item
    For each study, indicate whether it was observational or
    experimental, and note key characteristics of study populations or
    experimental models.
  \end{itemize}
\item
  Explain the scientific background and rationale for the investigation
  being reported.

  \begin{itemize}
  \tightlist
  \item
    Identify knowledge gaps addressed by the current study.
  \end{itemize}
\item
  Provide context for the study: include information on exposures and
  outcomes, and why they are relevant to environmental health.
\end{itemize}

\subsection{Objectives}\label{sec-objectives}

\begin{itemize}
\tightlist
\item
  Provide a clear description of the study hypotheses/aims/objectives,
  and eventually an overview of the approach used to address them.
\end{itemize}

\section{Methods}\label{sec-methods}

\subsection{Study design}\label{sec-design}

\begin{itemize}
\tightlist
\item
  Present key elements of study design
\end{itemize}

\subsection{Setting}\label{sec-setting}

\begin{itemize}
\tightlist
\item
  Describe the setting, locations, and relevant dates, including periods
  of recruitment, exposure, follow-up, and data collection.
\end{itemize}

\subsection{Participants}\label{sec-participants}

\begin{itemize}
\tightlist
\item
  Cohort study: eligibility criteria, and the sources and methods of
  selection of participants. Describe methods of follow-up.
\item
  Cross-sectional study: give eligibility criteria, and the sources and
  methods of selection of participants.
\item
  Describe informed consent protocols.
\item
  Report how and by whom \emph{race} or \emph{ethnicity} was defined,
  and why this information was included in the study design.
  Disaggregate race and ethnicity data to the fullest extent possible.
\end{itemize}

\subsection{Variables}\label{sec-vars}

\begin{itemize}
\tightlist
\item
  Clearly define all outcomes, exposures, predictors, potential
  confounders, and effect modifiers.
\item
  Explain the rationale for treating race as an exposure, confounder,
  effect modifier, or other type of variable in analyses.
\end{itemize}

\subsubsection{Confounders}\label{sec-confounders}

\subsubsection{Endocrine disrupting chemicals}\label{sec-edcs}

\subsubsection{Corticosteroids}\label{sec-steroids}

\subsubsection{Neurodevelopment}\label{sec-neurodevelopment}

\subsection{Data sources and measurement}\label{sec-dat-sources}

\begin{itemize}
\tightlist
\item
  For each variable of interest, give sources of data and details of
  methods of assessment (measurement).
\end{itemize}

\subsection{Bias}\label{sec-bias}

\begin{itemize}
\tightlist
\item
  Describe any efforts to address potential sources of bias.
\end{itemize}

\subsection{Study size}\label{sec-size}

\begin{itemize}
\tightlist
\item
  Explain how the study size was arrived at.
\end{itemize}

\subsection{Quantitative variables}\label{sec-quant-vars}

\begin{itemize}
\tightlist
\item
  Explain how quantitative variables were handled in the analyses. If
  applicable, describe which groupings were chosen and why.
\end{itemize}

\subsection{Statistical methods}\label{sec-stat-methods}

\begin{itemize}
\tightlist
\item
  Methods for selecting potential confounders (provide DAGs).
\item
  Describe all statistical methods with assumptions, including those
  used to control for confounding.

  \begin{itemize}
  \tightlist
  \item
    Description of outcome model, weighting method, estimand, and
    balance assessment.
  \item
    Description of method used to estimate effects (e.g.,
    g-computation).
  \item
    Description of method used for SE and CI.
  \end{itemize}
\item
  Describe any methods used to examine subgroups and interactions
  (sub-group analysis or moderation analysis or analysis of
  effect-modification).
\item
  Explain how missing data were addressed.
\item
  Cohort study: explain how loss to follow-up was addressed.
\item
  Cross-sectional study: describe analytical methods taking account of
  sampling strategy.
\item
  Describe any sensitivity analyses.
\item
  When referring to previous publications for methods' details, include
  a brief description of the approach, key assumptions and limitations,
  and any deviation.
\item
  Names and version numbers for the used software packages, including
  non-data arguments if deviating from the default ones.
\end{itemize}

\section{Results}\label{sec-res}

output/paper/tables.qmd@tbl-chem-info

\subsection{Participants}\label{sec-res-participants}

\begin{itemize}
\tightlist
\item
  Give reasons for non-participation at each stage.
\end{itemize}

\subsection{Descriptive data}\label{sec-res-pop-desc}

\subsection{Outcome data}\label{sec-res-out-desc}

\subsection{Main results}\label{sec-res-main}

\begin{itemize}
\tightlist
\item
  All results on which study conclusions or inferences are based,
  including null findings and results of secondary or sensitivity
  analyses, must be reported. Use of sub-headings that describe the
  nature of the results (but no declarative statements).

  \begin{itemize}
  \tightlist
  \item
    Provide a clear and concise description of all findings without
    extrapolating beyond the study results.
  \item
    Do not limit results to those \emph{statistically significant} or
    that support the study hypotheses. Avoid using statistical
    significance testing as the sole or primary criterion for
    interpreting the obtained results. If significance testing or
    \emph{p}-values are used, report numeric \emph{p}-values, rounded to
    1-2 digits, for all results.
  \end{itemize}
\item
  Give unadjusted and confounder-adjusted estimates and their precision
  (e.g., \(95\%\) confidence interval). Make clear which confounders
  were adjusted for and why they were included. Include the number of
  observations for each analysis after accounting for missing data.
  Include numeric data within figures (e.g., forest plots), or provide
  tables with corresponding numeric data for all figures.

  \begin{itemize}
  \tightlist
  \item
    \href{https://vincentarelbundock.github.io/marginaleffects/articles/tables.html}{\texttt{marginaleffects}
    tables}.
  \end{itemize}
\item
  Report category boundaries when continuous variables were categorized.
\end{itemize}

\subsection{Other analyses}\label{sec-res-other}

\begin{itemize}
\tightlist
\item
  Report other analyses done (e.g., analyses of subgroups and
  interactions, and sensitivity analyses).
\end{itemize}

\section{Discussion}\label{sec-discussion}

\subsection{Key results}\label{sec-disc-res-key}

\begin{itemize}
\tightlist
\item
  Summarise key results with reference to study objectives.
\item
  Provide a review of the relevant literature to put the study findings
  into context.

  \begin{itemize}
  \tightlist
  \item
    It should be complete and balanced, including inconsistent results.
  \item
    It should include, for each source, sufficient details: study
    design, sample size, population, specific exposures and outcomes.
  \end{itemize}
\end{itemize}

\subsection{Limitations}\label{sec-limitations}

\begin{itemize}
\tightlist
\item
  Discuss limitations of the study, taking into account sources of
  potential bias or imprecision.
\item
  Discuss both direction and magnitude of any potential bias.
\end{itemize}

Some limitations:

\begin{itemize}
\tightlist
\item
  Cross-sectional study.
\item
  Chemicals measured in night and morning samples, whereas metabolites
  (the outcome) were measured only in night samples.
\item
  Cortisol measured at night, when should be lowest.
\item
  Change of estimand when trimming weights.
\item
  Model misspecification.
\item
  Mixtures effect.
\item
  Residual confounding.
\item
  Some confounders were not used since large percentage of missing
  values.
\item
  Multiple comparisons.
\end{itemize}

\subsection{Interpretation}\label{sec-interpretation}

\begin{itemize}
\tightlist
\item
  End with a summary of the key findings and their implications for the
  study hypotheses, future research, and policy.
\item
  Give a cautious overall interpretation of results considering
  objectives, limitations, multiplicity of analyses, results from
  similar studies, and other relevant evidence.
\end{itemize}

\subsection{Generalisability}\label{sec-general}

\begin{itemize}
\tightlist
\item
  Discuss the generalisability (external validity) of the study results.
\end{itemize}

\newpage

\section{Funding}\label{funding}

\begin{itemize}
\tightlist
\item
  Give the source of funding and the role of the funders for the present
  study and, if applicable, for the original study on which the present
  article is based.
\end{itemize}

\newpage

\section*{References}\label{references}
\addcontentsline{toc}{section}{References}



\end{document}
